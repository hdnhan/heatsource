\documentclass[]{article}
% packages
\usepackage[utf8]{vietnam}
\usepackage{amsmath, amssymb, amsthm}
\usepackage{color, graphicx, cases}
\usepackage{hyperref}
\hypersetup{
	colorlinks=true,
	linkcolor=black,
	filecolor=black,      
	urlcolor=black,
}
\usepackage{array, multirow, booktabs}
\usepackage{caption, subcaption}
\usepackage{ragged2e} % using justifying
\numberwithin{equation}{section}
\everymath{\displaystyle}

%\usepackage{fancyhdr}
%\pagestyle{fancy}
%\lhead{HỒ ĐỨC NHÂN}
%\chead{}
%\rhead{ĐỒ ÁN TỐT NGHIỆP}
%\lfoot{}
%\cfoot{\thepage}
%\rfoot{}

\usepackage[a4paper,left=35mm,top=31mm,right=20mm,bottom=30mm]{geometry}
\renewcommand{\baselinestretch}{1.5}

% new definitions
\newtheorem{dl}{Định lý}[section]
\newtheorem{md}{Mệnh đề}[section]
\newtheorem{hq}{Hệ quả}[section]

\theoremstyle{definition}
\newtheorem{dn}{Định nghĩa}[section]
\newtheorem{vd}{Ví dụ}[section]
\newtheorem{cy}{Chú ý}[section]
\newtheorem{bt}{Bài toán}[section]
\newtheorem{nx}{Nhận xét}[section]

% reference
\usepackage{cleveref}
\crefname{dl}{\textbf{Định lý}}{}
\crefname{md}{\textbf{Mệnh đề}}{}
\crefname{hq}{\textbf{Hệ quả}}{}

\crefname{dn}{\textbf{Định nghĩa}}{}
\crefname{vd}{\textbf{Ví dụ}}{}
\crefname{cy}{\textbf{Chú ý}}{}
\crefname{bt}{\textbf{Bài toán}}{}
\crefname{nx}{\textbf{Nhận xét}}{}
\everymath{\displaystyle}

\title{Reconstruction of the right-hand side in parabolic equations from integral observations}

\author{Ta Thi Thanh Mai\thanks{email: mai.tathithanh@hust.edu.vn} \and Ho Duc Nhan\thanks{email: hdnhan28@gmail.com}}

\thanksmarkseries{arabic}
\date{\footnotesize\textit{School of Applied Mathematics and Informatics, Hanoi University of Science and Technology, \\ No.1 Dai Co Viet street, Hai Ba Trung District, Hanoi, Vietnam}}

\begin{document}
\justifying
\maketitle
%\thispagestyle{empty}
%\normalsize
\begin{abstract}
We study the problem of determining the right-hand side in
parabolic equations with time-dependent coefficients from integral observations which can be regarded as generalizations of point-wise interior observations. 
%check lai
%We develop the previous research methods for inverse problems 
Our approach is new in the sense that for determining the right-hand side we do not assume that the data available in the whole space domain or in a subset of the space domain, but some integral observations during the time interval. 
We propose a variational method in combination with Tikhonov regularization for solving the problem and then discrete it by space-time finite element methods. As a result, a numerical scheme with error estimate are given.
The discretized minimization problem is solved by the conjugate gradient method. Some numerical examples are investigated for showing the efficiency and accuracy
of present method.
\end{abstract}

\textbf{Keywords:} Inverse source problems, least squares method, Tikhonov regularization, space-time finite element method, conjugate gradient method.
\section{Introduction and Problem setting}
Over the last decades, the problem of determining a term in the right hand side of parabolic equations has received a large amount of attention from both engineers and mathematicians. Despite a vast literature on the existence, uniqueness and stability estimates of a solution to the problem, challenging topics still require further investigation to reach the ill-posedness and possible nonlinearity.
For surveys on the subject, we refer the reader to the books [5, 23, 29, 30, 43] and the recent paper [48].\\
To be more detailed, let $\Omega \subset \mathbb{R}^d,\, d\in \mathbb{N^+}$ be a bounded domain with a boundary $\Gamma$ and endow the cylinder $Q=\Omega\times (0,\, T]$ and lateral surface area $S=\Gamma \times (0,\, T]$ where $T>0$. 
\\
Consider the unsteady heat equation:
\begin{align}\label{1.1}
	\frac{\partial u}{\partial t}-\sum_{i, j=1}^{d}\frac{\partial}{\partial x_j}\left(a_{ji}(x, t)\frac{\partial u}{\partial x_i}\right)=F(x, t), \quad(x, t)\in Q,
\end{align}
with the initial and Dirichlet conditions, respectively
\begin{align}
	u(x, 0)&=u_0(x),\quad x\in \Omega,\label{1.2}\\
	u(x, t)&=0,\quad(x, t)\in S, \label{1.3}
\end{align}
where
\begin{align*}
	&a_{ij}\in L^{\infty}(Q),\, a_{ij}=a_{ji},\; \forall i, j\in \{1, 2, ..., d\},\\
	&\lambda_1\left\|\xi\right\|^2\leq \sum_{i, j=1}^{d}a_{ij}\xi_i\xi_j\leq \lambda_2\left\|\xi\right\|^2,\; \forall \xi\in\mathbb{R}^d,\\
	&u_0\in H^1_0(\Omega),\; F\in L^2(0,\, T;\, H^{-1}(\Omega)),
\end{align*}
with $\lambda_1$ và $\lambda_2$ are positive constants.
\\
The direct problem is to determine $u$ when all data $a_{ji}, i, j=\overline{1, d}, \,u_0, \,\varphi$ and $F$ in \cref{1.1,1.2,1.3} are given. On the other hand, the inverse problem (IP) is to identify a missed data such as the right hand side $F$ when some additional observations on the solution $u$ are available. 
\\
We consider the right hand side of the equation \eqref{1.1} following the form $F(x, t)=f(.)q(x, t)+g(x, t)$, where $q(x, t),\, g(x, t)$ are given and $f(.)$ can be either $f(x, t)$, $f(x)$ or $f(t)$.
%Denote $N_m$ is the number of measurements and $\ell_k u(f) =\omega_k, k=\overline{1, N_m}$ is the value of the $kth$ measurement. 
We have different inverse problems depending on either the form of $F$ or the observation on the solution $u$: 
\begin{itemize}
	\item IP1: Find $f(x, t)$ if $u(x, t)$ is given on $Q$ \cite{a1, a2}.
	\item IP2: Find $f(x)$ if $u(x, T)$ is given on $\Omega$ \cite{a3, a4, a5, a6}.
	\item IP3: Find $f(t)$ if $\int_\Omega w(x)u(x, t)dx$ and $w(x)>0, \forall x\in \Omega$ are given \cite{a7, a8, a9}. This observation called \textit{integral observation}. Furthermore, an observation derives from integral observation called \textit{point observation} if $w(x)$ is a dirac delta function $\delta(x-x_0)$, so that $\int_\Omega\delta(x-x_0)u(x, t)dx=u(x_0, t), x_0$ is a point in $\Omega$ \cite{a10, a11, a12}. Beside that, find $f(x, t)$ or $f(x)$ if some integral or point observations are available \cite{a13}.
	%\item IP3: Find $f(t)$ if $\ell_ku=\int_\Omega w_k(x)u(x, t)dx=\omega_k(t)$ and $w_k(x)>0, \forall x\in \Omega$ are given. This observation called \textit{integral observation}. Furthermore, an observation derives from integral observation called \textit{point observation} if $w_k(x)$ is a dirac delta function $\delta_k(x-x_k)$, so that $\ell_k(t)=\int_\Omega\delta_k(x-x_k)u(x, t)dx=u(x_k, t)=\omega_k(t)$. 
\end{itemize}
Donate $w$ is the value of the observation that given and $\ell u(f)$ is the result of the observation based on the solution $u$ we get. In this paper, we present  the case of having many integral observations with $N_m$ is the number of observations, others can be proved similarly. So, to solve this problem, we need to minimize the least square functional \cite{a14, a15}
$$J_{\gamma}(f)=\frac{1}{2}\sum_{k=1}^{N_{m}}\left\|\ell_k u(f)-\omega_k\right\|_{L^2(0, T)}^2.$$
However, this minimization problem is unstable and there might be many minimizers to it. Therefore, we minimize the Tikhonov functional instead
$$J_{\gamma}(f)=\frac{1}{2}\sum_{k=1}^{N_{o}}\left\|\ell_k u(f)-\omega_k\right\|_{L^2(0, T)}^2+\frac{\gamma}{2}\left\|f-f^*\right\|_{*}^2,$$
with $\gamma>0$ being a regularization parameter, $f^*$ is an a prior estimation of $f$ and $\left\|.\right\|_*$ an appropriate norm.

This paper is organized into five sections. The next section provides a variational problem and prove that the Tikhonov functional is Frechet differentiable. Section \ref{section3} reviews some classical theories of conjugate gradient method, as well as the computation of descent direction. Section \ref{section4} presents the numerical method for solving the state and adjoint problem by space-time finite element methods; the convergence results is outlined in Section .... All the numerical test cases are presented in Section \ref{section5}. The last section gives some perspectives and comments about the effectiveness and limitations of the present scheme.

\section{Variational problem}\label{section2}
To introduce the concept of weak form, we use the standard Sobolev spaces $H^1(\Omega),\, H^1_0(\Omega),\, H^{1, 0}(Q)$ and $H^{1, 1}(Q)$ \cite{b1, b2, b3}. Further, for a Banach space $B$, we define
$$L^2(0, T; B)=\left\{u:u(t)\in B \text{ a.e } t\in (0, T) \text{ and } \left\|u\right\|_{L^2(0, T;\; B)} <\infty \right\},$$
with the norm
$$\left\|u\right\|_{L^2(0, T; B)}^2=\int_0^T\left\|u(t)\right\|^2_Bdt.$$
In this paper, we will use an equivalent norm in $L^2\left(0, T; H^1_0(\Omega)\right)$ with the norm
$$\left\|u\right\|_{L^2(0, T;\; H^1_0(\Omega))}^2=\int_Q \sum_{i, j=1}^{d}a_{ji}\frac{\partial u}{\partial x_i}\frac{\partial u}{\partial x_j}dxdt.$$
So, with the duality pairing $\langle ., .\rangle_Q$, the dual norm will be
$$\left\|u\right\|_{L^2(0, T;\; H^{-1}(\Omega))}=\sup_{0\neq v\in L^2(0, T;\; H^1_0(\Omega))}\frac{\langle u, v\rangle_Q}{\left\|v\right\|_{L^2(0, T;\; H^1_0(\Omega))}}.$$
In the sequel, we shall use the space $W(0, T)$ define as
$$W(0, T)=\left\{u: u\in L^2(0, T; H^1_0(\Omega)), u_t\in L^2\left(0, T; H^{-1}(\Omega) \right)\right\}$$
A week solution in $W(0, T)$ of the problem \cref{1.1,1.2,1.3} is a function $u(x, t)\in W(0, T)$ satisfying the identity
\begin{align}\label{2.1}
	\int_{Q}\left[\frac{\partial u}{\partial t}v+\sum_{i, j=1}^{d}a_{ji}\frac{\partial u}{\partial x_i}\frac{\partial v}{\partial x_j}\right]dxdt=\int_{Q}Fvdxdt,\;\forall v \in L^2\left(0, T; H^1(\Omega)\right).
\end{align}
or 
\begin{align*}
	a(u, v)=\langle F, v\rangle_{Q},\;\forall v \in L^2\left(0, T; H^1_0(\Omega)\right).
\end{align*}
and 
\begin{align}\label{2.2}
	u(x, 0)=u_0,\; x\in \Omega.
\end{align}
From now on, we donate $X=\left\{u: u\in W(0, T): u(x, 0)=0\right\}$ and $Y=L^2\left(0, T; H^1_0(\Omega)\right)$. Obviously, we have $X\subset Y$. We split $u(x, t)=\overline{u}(x, t)+\overline{u}_0$ for $(x, t)\in Q$ where $\overline{u}_0\in W(0, T)$ is some extension of the given initial datum $u_0\in H^1_0(\Omega)$. Here, we use space-time finite element method \cite{a16} and therefore we can prove that there exists a unique solution $\overline{u}\in X$ of the problem \cref{1.1,1.2,1.3} that satisfies 
\begin{align}\label{2.3}
	\left\|\overline{u}\right\|_{W(0, T)} \leq c_d \left(\left\|F\right\|_{L^2\left(0, T; H^{-1}(\Omega)\right)}+\left\|\overline{u}_0\right\|_{W(0, T)}\right).
\end{align}
We suppose that $F$ has the form $F(x, t)=f(x, t)q(x, t)+g(x, t)$ with \color{red} $f\in L^2(Q),\, q\in L^\infty(Q)$ and $g\in L^2(Q)$\color{black}. We hope to recover $f(x, t)$ from the observation. Since the solution $u(x, t)$ depends on the function $f(x, t)$, so we denote it by $u(x, t, f)$ or $u(f)$. Identify $f(x, t)$ satisfying 
$$\ell_k u(f)=\omega_k,\; \forall k = \overline{1, N_m}.$$
%where $\ell_k u(f)$ is the observation on the solution depending on $f$. We suppose to solve (IP2) problem and (IP1) will be done the same way. 
We need to minimize the Tikhonov functional
\begin{align}\label{2.4}
	J_{\gamma}(f)=\frac{1}{2}\sum_{k=1}^{N_m}\left\|\ell_k u(f)-\omega_k\right\|_{L^2(0, T)}^2+\frac{\gamma}{2}\left\|f-f^*\right\|_{L^2(Q)}^2.
\end{align}
We will prove that $J_\gamma$ is Frechet differentiable and drive a formula for its gradient. In doing so, we need the adjoint problem
\begin{align}\label{2.5} 
	\begin{cases}
		-\frac{\partial p}{\partial t}-\sum_{i, j=1}^{d}\frac{\partial}{\partial x_j}\left(a_{ji}(x, t)\frac{\partial p}{\partial x_i}\right)=\sum_{k=1}^{N_m}w(x)\left(\ell_k u(f)-\omega_k\right), & (x, t)\in Q,\\
		u(x, t)=0, & (x, t)\in S\\
		p(x, T)=0, & x\in \Omega.
	\end{cases}
\end{align}
By changing the time direction, meaning $\tilde{p}(x, t)=p(x, T-t)$, we will get a Dirichlet problem for parabolic equations.
\begin{dl}
	The functional $J_\gamma$ is Frechet differentiable and its gradient $\nabla J_\gamma$ at $f$ has the form 
	\begin{align}\label{2.6}
		\nabla J_\gamma(f)=q(x, t)p(x, t)+\gamma \left(f(x, t)-f^*(x, t)\right)
	\end{align}
\end{dl}
\begin{proof}
	By taking a small variation $\delta f \in L^2(Q)$ of $f$ and denoting $\delta u(f)=u(f+\delta f)-u(f)$, we have
	\begin{align*}
		J_0(f+\delta f)-J_0(f)&=\frac{1}{2}\sum_{k=1}^{N_m}\left\|\ell_k u(f+\delta f)-\omega_k\right\|^2_{L^2(0, T)}-\frac{1}{2}\sum_{k=1}^{N_m}\left\|\ell_k u(f)-\omega_k\right\|^2_{L^2(0, T)}\\
		&=\frac{1}{2}\sum_{k=1}^{N_m}\left\|\ell_k \delta u(f) +\ell_k u(f)-\omega_k\right\|^2_{L^2(0, T)}-\frac{1}{2}\sum_{k=1}^{N_m}\left\|\ell_k u(f)-\omega_k\right\|^2_{L^2(0, T)}\\
		&=\sum_{k=1}^{N_m}\frac{1}{2}\left\|\ell_k \delta u(f)\right\|^2_{L^2(0, T)}+\sum_{k=1}^{N_m}\left\langle \ell_k \delta u(f), \ell_k u(f)-\omega_k\right\rangle_{L^2(0, T)},
	\end{align*}
	where $\delta u(f)$ is the solution to this problem
	\begin{align}\label{2.7}
		\begin{cases}
			\frac{\partial \delta u}{\partial t}-\sum\limits_{i, j=1}^{d}\frac{\partial}{\partial x_j}\left(a_{ji}(x, t)\frac{\partial \delta u}{\partial x_i}\right)=q(x, t)\delta f,&(x, t)\in Q,\\
			\delta u(x, t)=0, & (x, t)\in S,\\
			\delta u(x, 0)=0, &x\in \Omega.
		\end{cases}
	\end{align}
	Because the priori estimate \eqref{2.3} for the direct problem, we have
	\begin{align}\label{2.8}
		\left\|\ell_k\delta u(f)\right\|_{L^2(0, T)}^2=o\left(\left\|\delta f\right\|_{L^2(Q)}\right)\, \text{when } \left\|\delta f\right\|_{L^2(Q)}\to 0.
	\end{align}
	What is more, applying the Green formula [...] for \eqref{2.5} and \eqref{2.7}, we get
	\begin{align}\label{2.9}
		\sum_{k=1}^{N_m}\int_{Q} \delta u(x, t) w(x) \left(\ell_k u(f)-\omega_k(t)\right)dxdt=\int_{Q} p(x, t)q(x, t)\delta f(x, t)dxdt
	\end{align}
	According to \eqref{2.8} and \eqref{2.9}, we obtain
	\begin{align*}
		J_0(f+\delta f)-J_0(f)&=\sum_{k=1}^{N_m}\int_{Q}\delta u(x, t)w(x)\left(\ell_k u(f)-\omega_k(t)\right)ds+o\left(\left\|\delta f\right\|_{L^2(Q)}\right)\notag\\
		&=\int_Q q(x, t)p(x, t)\delta f(x, t)dxdt+o\left(\left\|\delta f\right\|_{L^2(I)}\right)\notag\\
		&=\left\langle qp,\delta f \right\rangle_{L^2(Q)}+o\left(\left\|\delta f\right\|^2_{L(Q)}\right).
	\end{align*}
	Therefore, we will obtain
	$$J_\gamma(f+\delta f)-J_\gamma(f)=\left\langle qp,\delta f \right\rangle_{L^2(Q)}+\gamma\left\langle f-f^*,\delta f \right\rangle_{L^2(Q)}+o\left(\left\|\delta f\right\|^2_{L(Q)}\right).$$
	Hence the functional $J_\gamma$ is Frechet differentiable and its gradient $\nabla J_\gamma$ at $f$ has the form \eqref{2.6}. The theorem is proved.
\end{proof}
\begin{cy}
	In this theorem, we write the Tikhonov functional for $F(x, t)=f(x, t)q(x, t)+g(x, t)$. But when F has another form, the penalty term should be modified
	\begin{itemize}
		\item $F(x, t)=f(x)q(x, t)+g(x, t)$: the penalty functional is $\left\|f-f^*\right\|_{L^2(\Omega)}$ and $$\nabla J_0(f)=\int_0^Tq(x, t)p(x, t)dt.$$
		\item $F(x, t)=f(t)q(x, t)+g(x, t)$: the penalty functional is $\left\|f-f^*\right\|_{L^2(0, T)}$ and $$\nabla J_0(f)=\int_\Omega q(x, t)p(x, t)dt.$$
	\end{itemize}
\end{cy}

\section{Conjugate gradient method}\label{section3}
\noindent To find $f$ satisfied \eqref{2.4}, we use the conjugate gradient method (CG). Its iteration follows, we assume that at the $k$th iteration, we have $f^k$ and then the next iteration will be
$$f^{k+1}=f^k+\alpha_kd^k,$$
with
\begin{align*}
	d^k&=\left\{\begin{array}{ll}
	-\nabla J_\gamma(f^k),& k=0,\\
	-\nabla J_\gamma(f^k)+\beta_kd^{k-1},& k>0,
	\end{array}\right.\\\\
	\beta_k&=\frac{\left\|\nabla J_\gamma (f^k)\right\|^2_{L^2(I)}}{\left\|\nabla J_\gamma (f^{k-1})\right\|^2_{L^2(I)}},
\end{align*}
and
$$\alpha_k=\operatorname*{arg\,min}_{\alpha\geq 0}J_\gamma(f^k+\alpha d^k).$$
To identify $\alpha_k$, we consider two problems
\begin{bt}\label{bt2.1}
	Denote the solution of this problem is $u[f]$
	\begin{align*}
		\begin{cases}
			\frac{\partial u}{\partial t}-\sum_{i, j=1}^{d}\frac{\partial}{\partial x_j}\left(a_{ji}(x, t)\frac{\partial u}{\partial x_i}\right)=f(x, t)q(x, t),&(x, t)\in Q,\\
			u(x, t)=0, & (x, t)\in S,\\
			u(x, 0)=0,&x\in \Omega.
		\end{cases}
	\end{align*}
\end{bt}
\begin{bt}\label{bt2.2}
	Denote the solution of this problem is $u(u_0, \varphi)$
	\begin{align*}
		\begin{cases}
			\frac{\partial u}{\partial t}-\sum_{i, j=1}^{d}\frac{\partial}{\partial x_j}\left(a_{ji}(x, t)\frac{\partial u}{\partial x_i}\right)=g(x, t),&(x, t)\in Q,\\
			u(x, t)=0, & (x, t)\in S,\\
			u(x, 0)=u_0(x),&x\in \Omega.
		\end{cases}
	\end{align*}
\end{bt}
\noindent If we do so, the observation operators have the form $\ell_i u(f)=\ell_i u[f]+\ell_i u(u_0, \varphi)=A_if+\ell_i u(u_0, \varphi)$, with $A_i$ being bounded linear operators from $L^2(Q)$ to $L^2(0, T)$.\\
We have
\begin{align*}
	J_{\gamma}(f^k+\alpha d^k)&=\frac{1}{2}\sum_{i=1}^{N_m}\left\|\ell_i u(f^k+\alpha d^k)-\omega_i\right\|_{L^2(0, T)}^2+\frac{\gamma}{2}\left\|f^k+\alpha d^k-f^*\right\|_{L^2(Q)}^2\\[0.2cm]
	&=\frac{1}{2}\sum_{i=1}^{N_m}\left\|\alpha A_id^k+A_if^k+\ell_i u(u_0, \varphi)-\omega_i\right\|_{L^2(0, T)}^2+\frac{\gamma}{2}\left\|f^k+\alpha d^k-f^*\right\|_{L^2(Q)}^2\\[0.2cm]
	&=\frac{1}{2}\sum_{i=1}^{N_m}\left\|\alpha A_id^k+\ell_i u(f^k)-\omega_i\right\|_{L^2(0, T)}^2+\frac{\gamma}{2}\left\|f^k+\alpha d^k-f^*\right\|_{L^2(Q)}^2.
\end{align*}
Differentiating $J_\gamma(f^k+\alpha d^k)$ with respect to $\alpha$, we get
\begin{align*}
	\frac{\partial J_\gamma(f^k+\alpha d^k)}{\partial \alpha} &= \alpha\sum_{i=1}^{N_m}\left\|A_id^k \right\|_{L^2(0, T)}^2+\sum_{i=1}^{N_m}\left\langle A_id^k,\ell_i u(f^k)-\omega_i\right\rangle_{L^2(0, T)}\\[0.2cm]
	&\quad+\gamma\alpha\left\| d^k\right\|_{L^2(Q)}^2+\gamma\left\langle d^k, f^k-f^*\right\rangle_{L^2(Q)}.
\end{align*}
Putting $\frac{\partial J_\gamma(f^k+\alpha d^k)}{\partial \alpha}=0$, we obtain
\begin{align*}
	\alpha_k&=-\frac{\displaystyle\sum_{i=1}^{N_m}\left\langle A_id^k, \ell_i u(f^k)-\omega_i\right\rangle_{L^2(0, T)}+\gamma\left\langle d^k, f^k-f^*\right\rangle_{L^2(Q)}}{\displaystyle\sum_{i=1}^{N_m}\left\|A_id^k\right\|^2_{L^2(0, T)}+\gamma\left\|d^k\right\|^2_{L^2(Q)}}\\[0.2cm]
	&=-\frac{\displaystyle\sum_{i=1}^{N_m}\left\langle d^k, A_i^*\left(\ell_i u(f^k)-\omega_i\right)\right\rangle_{L^2(Q)}+\gamma\left\langle d^k, f^k-f^*\right\rangle_{L^2(Q)}}{\displaystyle\sum_{i=1}^{N_m}\left\|A_id^k\right\|^2_{L^2(0, T)}+\gamma\left\|d^k\right\|^2_{L^2(Q)}}\\[0.2cm]
	&=-\frac{\displaystyle\sum_{i=1}^{N_m}\left\langle d^k, A_i^*\left(\ell_i u(f^k)-\omega_i\right)+\gamma(f^k-f^*)\right\rangle_{L^2(Q)}}{\displaystyle\sum_{i=1}^{N_m}\left\|A_id^k\right\|^2_{L^2(0, T)}+\gamma\left\|d^k\right\|^2_{L^2(Q)}}\\[0.2cm]
	&=-\frac{\left\langle d^k,\nabla J_\gamma(f^k)\right\rangle_{L^2(Q)}}{\displaystyle\sum_{i=1}^{N_m}\left\|A_id^k\right\|^2_{L^2(0, T)}+\gamma\left\|d^k\right\|^2_{L^2(Q)}}.
\end{align*}
Because of $d^k=r^k+\beta_kd^{k-1},\, r^k=-\nabla J_\gamma (f^k)$ and $\left\langle r^k,d^{k-1}\right\rangle_{L^2(I)}=0$, we get 
$$\alpha_k=\frac{\left\|r^k\right\|^2_{L^2(Q)}}{\displaystyle\sum_{i=1}^{N_m}\left\|A_id^k\right\|^2_{L^2(0, T)}+\gamma\left\|d^k\right\|^2_{L^2(Q)}}.$$
Thus, the CG algorithm is set up by following loop:

\noindent \textbf{CG algorithm}
\begin{itemize}
	\item[1.] Set $k=0$, initiate $f^0$.
	\item[2.] For $k=0, 1, 2,...$. Calculate
	$$r^k=-\nabla J_\gamma(f^k).$$
	Update\\
	\begin{align*}
		d^k&=\left\{\begin{array}{ll}
		r^k,& k=0,\\
		r^k+\beta_kd^{k-1},& k>0,
		\end{array}\right.\\\\
		\beta_k&=\frac{\left\|r^k\right\|^2_{L^2(Q)}}{\left\|r^{k-1}\right\|^2_{L^2(Q)}}.
	\end{align*}
	\item[3.] Calculate
	$$\alpha_k=\frac{\left\|r^k\right\|^2_{L^2(Q)}}{\displaystyle\sum_{i=1}^{N_m}\left\|A_id^k\right\|^2_{L^2(0, T)}+\gamma\left\|d^k\right\|^2_{L^2(Q)}}.$$
	Update
	$$f^{k+1}=f^{k}+\alpha_kd^k.$$
\end{itemize}

\section{Finite element method}\label{section4}
We rewrite the Tikhonov functional
\begin{align*}
	J_\gamma(f)&=\frac{1}{2}\sum_{i=1}^{N_m}\left\|\ell_i u[f]+\ell_i u(u_0, \varphi)-\omega_i\right\|^2_{L^2(0, T)}+\frac{\gamma}{2}\left\|f-f^*\right\|^2_{L^2(Q)}\\
	&=\frac{1}{2}\sum_{i=1}^{N_m}\left\|A_if+\ell_i u(u_0, \varphi)-\omega_i\right\|^2_{L^2(0, T)}+\frac{\gamma}{2}\left\|f-f^*\right\|^2_{L^2(Q)}\\
	&=\frac{1}{2}\sum_{i=1}^{N_m}\left\|A_if-\hat{\omega}_i\right\|^2_{L^2(0, T)}+\frac{\gamma}{2}\left\|f-f^*\right\|^2_{L^2(Q)},
\end{align*}
with $\hat{\omega}_i=\omega_i-\ell_i u(u_0, \varphi)$.
\\
The solution $f^\gamma$ of the minimization problem \eqref{2.4} is characterized by the first-order optimality condition
\begin{align}\label{3.1}
	\nabla J_\gamma(f^\gamma)= \sum_{i=1}^{N_m}A^*_i(A_if^\gamma-\hat{\omega}_i)+\gamma(f^\gamma-f^*)=0,
\end{align}
with $A_i^*: L^2(0, T)\to L^2(Q)$ is the adjoint operator of $A_i$ defined by $\sum_{i=1}^{N_m}A_i^*\left(\ell_i u(f) - \omega_i\right) = p$ where $p$ is the solution of the adjoint problem \eqref{2.5}. 
\\
We will approximate \eqref{3.1} by space-time finite element method. In fact, we will approximate $A_k$ and $A^*_k$ as follows.

\subsection{Approximation of $A_k,\, A_k^*$}
We suppose that finite spaces $W_h\subset W(0, T)$, $X_h \subset X$ and $Y_h \subset Y$, we assume that $X_h \subset Y_h$. 
The Galerkin-Petrov discretization of the variational problem \eqref{2.1} is to find $\overline{u}_h\in X_h$ such that
\begin{align}\label{3.2}
	a(\overline{u}_h, v_h)=\langle F, v_h\rangle_Q-a(\overline{u}_0, v_h), \forall v_h\in Y_h.
\end{align}
For the space-time domain $Q=\Omega\times I\subset \mathbb{R}^{d+1}$, we consider a sequence of admissible decompositions $Q_h$ into shape regular simplicity finite element $q_l$
$$Q_h=\cup_{l=1}^{N}\bar{q}_l.$$
Denote $\left\lbrace (x_k, t_k)\right\rbrace_{k=1}^M $ is a set of nodes $(x_k, t_k)\in \mathbb{R}^{d+1}$. We introduce a reference element $q\in \mathbb{R}^{d+1}$ which any element $q_l$ can maple to $q$ by using
$$\begin{pmatrix}x\\t\end{pmatrix}=\begin{pmatrix}x_k\\t_k\end{pmatrix}+J_l\begin{pmatrix}\xi\\\tau\end{pmatrix}, \; \begin{pmatrix}\xi\\\tau\end{pmatrix}\in q.$$
with $\Delta_l$ is the volume of $q_l$ 
$$\Delta_l=\int_{q_l}dxdt=\det J_l\int_q d\xi d\tau=|q|\det J_l,$$
and the local mesh width
$$h_l=\Delta_l^{\frac{1}{d+1}},\; h:=\max_{l=1, ..., N}h_l.$$
Note that
$$|q|=\begin{cases}
	\frac{1}{2},& d=1,\\[0.1cm]
	\frac{1}{6},& d=2.
\end{cases}$$
The discrete variational problem \eqref{3.2} admits a unique solution $\overline{u}_h\in X_h$. Let $u_h=\overline{u}_h+\overline{u}_{0,h}\in W_h$. Hence, the discrete version of the optimal control problem \eqref{2.4} will be
$$J_{\gamma, h}(f)=\frac{1}{2}\sum_{i=1}^{N_m}\left\|A_{i, h}f-\hat{\omega}_{i, h}\right\|^2_{L^2(0, T)}+\frac{\gamma}{2}\left\|f-f^*\right\|^2_{L^2(Q)}\to \min.$$
Let $f^\gamma_h$ be the solution of this problem is characterized by the variational equation
\begin{align}\label{3.3}
	\nabla J_{\gamma, h}(f^\gamma_h)= \sum_{i=1}^{N_m}A_{i, h}^*(A_{i, h}f^\gamma-\hat{\omega}_{i, h})+\gamma(f^\gamma_h-f^*)=0,
\end{align}
where $A_{i, h}^*$ is the adjoint operator of $A_{i, h}$. But it is hardly to find $A^*_{i, h}$ from $A_{i, h}$ in practice. So we define a proximate $\hat{A}_{i, h}^*$ of $A_i^*$ instead. In deed, we have $\sum_{i=1}^{N_m}\hat{A}^*_{i, h}\phi_i=p_h$, where $\phi_i=\ell_i u(f) - \omega_i$ and $p_h$ is the approximate solution of adjoint problem \eqref{2.5}. Therefore, the equation above will be
\begin{align}\label{3.4}
	\nabla J_{\gamma, h}(f^\gamma_h)\simeq\nabla J_{\gamma, h}(\hat{f}^\gamma_h)= \sum_{i=1}^{N_m}\hat{A}_{i, h}^*(A_{i, h}\hat{f}^\gamma-\hat{\omega}_{i, h})+\gamma(\hat{f}^\gamma_h-f^*)=0,
\end{align}
Moreover, the observation will have noise in practice, so instead of $\omega$, we only get $\omega^{\delta}$ satisfying
$$\left\| \omega-\omega^\delta\right\|_{L^2(S_1)}\leq \delta.$$
Therefore, instead of getting $\hat{f}^\gamma_h$ that satisfies the equation \eqref{3.5}, we will get $\hat{f}^{\gamma, \delta}_h$ satisfying
\begin{align}\label{3.5}
	\nabla J_{\gamma, h}\left(\hat{f}^{\gamma, \delta}_h\right)= \sum_{i=1}^{N_m}\hat{A}_{i, h}^*(A_{i, h}\hat{f}^{\gamma, \delta}_h-\hat{\omega}_{i, h}^\delta)+\gamma(\hat{f}^{\gamma, \delta}_h-f^*)=0,
\end{align}
with $\hat{\omega}_{i, h}^\delta=\omega^\delta-\ell_i u_h(u_0, \varphi)$.
\subsection{Convergence results}
\begin{dl}\label{dl3.2}
	Let $u(x, t)$ be the solution of variational problem \eqref{2.1} - \eqref{2.2} and $\overline{u}_h(x, t)$ be the solution for \eqref{3.2} and $u_h(x, t)=\overline{u}_h(x, t)+\overline{u}_{0, h}(x, t)$. Then there holds the error estimate
	\begin{align}\label{3.6}
		\left\| u-u_h\right\|_{L^2(0, T;\; H^1_0(\Omega))}\leq ch \left|\overline{u}\right|_{H^2(\Omega)}.
	\end{align}
	and 
	\begin{align}\label{3.7}
		\left\| u-u_h\right\|_{L^2(Q)}\leq ch^2 \left|u\right|_{H^2(\Omega)}.
	\end{align}
\end{dl}
\noindent What is more,
\begin{align*}
	\left\| \sum_{i=1}^{N_m}\left(A_i^*-\hat{A}^*_{i,h}\right)\phi_i\right\|_{L^2(Q)}^2=\int_Q (p-p_h)^2dxdt=\left\| p-p_h\right\|_{L^2(Q)}^2
\end{align*}
\begin{align}\label{3.8}
	\Rightarrow \left\| \sum_{i=1}^{N_m}\left(A_i^*-A^*_{i, h}\right)\phi_i\right\|_{L^2(Q)}\leq ch^2.
\end{align}
Let $u_h[f]$ và $u_h(u_0, \varphi)$ are the approximate solutions of \cref{bt2.1} and \cref{bt2.2} by using space-time finite element method. We define $A_h$ of $A$ is $A_hf=\ell u_h[f]$ and $\hat{\omega}_h=\omega-\ell u_h(u_0, \varphi)$. We have
\begin{align*}
	\left\| \sum_{i=1}^{N_m}\left(A_i-A_{i, h}\right)f\right\|_{L^2(0, T)}^2=\sum_{i=1}^{N_m}\left\| \ell_i u[f]-\ell_i u_h[f]\right\|_{L^2(0, T)}^2\leq\sum_{i=1}^{N_m} \left\|w_i\right\|^2_{L^2(\Omega)}\left\| u[f]-u_h[f]\right\|_{L^2(Q)}^2
\end{align*}
\begin{align}\label{3.9}
	\Rightarrow\left\| \sum_{i=1}^{N_m}\left(A_i-A_{i, h}\right)f\right\|_{L^2(0, T)}\leq ch^2
\end{align}
and 
\begin{align*}
	\left\|\sum_{i=1}^{N_m}\left(\hat{\omega}_i-\hat{\omega}_{i, h}\right)\right\|_{L^2(0, T)}^2=\sum_{i=1}^{N_m}\left\| \ell_iu(u_0,\varphi)-\ell_iu_h(u_0, \varphi)\right\|_{L^2(0, T)}^2\leq\sum_{i=1}^{N_m} \left\|w_i\right\|^2_{L^2(\Omega)}\left\| u(u_0,\varphi)-u_h(u_0, \varphi)\right\|_{L^2(Q)}^2
\end{align*}
\begin{align}\label{3.10}
	\Rightarrow\left\|\sum_{i=1}^{N_m}\left(\hat{\omega}_i-\hat{\omega}_{i, h}\right)\right\|_{L^2(0, T)}\leq ch^2
\end{align}

\begin{dl}\label{dl3.3}
	Let $f^\gamma$ and $\hat{f}^\gamma_h$ are the solution of variational problems \eqref{3.1} and \eqref{3.4}, respectively. Then there hold a error estimate
	\begin{align}\label{3.11}
	\left\|f^\gamma-\hat{f}^\gamma_h \right\|_{L^2(Q)}\leq ch^2.
	\end{align}
\end{dl}
\begin{proof} From equations \eqref{3.1} and \eqref{3.4}, we will have
	\begin{align*}
		\gamma \left(f^\gamma-\hat{f}^\gamma_h\right)&=\sum_{i=1}^{N_m}\hat{A}^*_{i, h}\left(A_{i, h}\hat{f}^\gamma_h-\hat{\omega}_{i, h}\right)-\sum_{i=1}^{N_m}A^*_i\left(A_if^\gamma-\hat{\omega}_i\right)\\
		&=\sum_{i=1}^{N_m}\left(\hat{A}^*_{i, h}-A^*_i\right)\left(A_{i, h}\hat{f}^\gamma_h-\hat{\omega}_{i, h}\right)+\sum_{i=1}^{N_m}A^*_iA_{i, h}\left(\hat{f}^\gamma_h-f^\gamma\right)\\
		&\quad+\sum_{i=1}^{N_m}A_i^*\left(A_{i, h}-A_i\right)f^\gamma+\sum_{i=1}^{N_m}A^*_i\left(\hat{\omega}_i-\hat{\omega}_{i, h}\right)
	\end{align*}
	According to \eqref{3.8}, \eqref{3.9} and \eqref{3.10}, we have
	\begin{align*}
		&\left\|\sum_{i=1}^{N_m} \left(\hat{A}^*_{i, h}-A^*_i\right)\left(A_{i, h}\hat{f}^\gamma_h-\hat{\omega}_{i, h}\right)\right\|_{L^2(0, T)}\leq ch^2,\\
		&\left\|\sum_{i=1}^{N_m} A^*_i\left(A_{i, h}-A_i\right)f^\gamma\right\|_{L^2(0, T)}\leq ch^2,\\
		&\left\|\sum_{i=1}^{N_m}A^*_i\left(\hat{\omega}_i-\hat{\omega}_{i, h}\right) \right\|_{L^2(I)}\leq ch^2.
	\end{align*}
	We take apart this
	$$\sum_{i=1}^{N_m}A^*_iA_{i, h}\left(\hat{f}^\gamma_h-f^\gamma\right)=\sum_{i=1}^{N_m}A^*_i\left(A_{i, h}-A_i\right)\left(\hat{f}^\gamma_h-f^\gamma\right)+\sum_{i=1}^{N_m}A^*_iA_i\left(\hat{f}^\gamma_h-f^\gamma\right).$$
	Moreover, we have
	\begin{align*}
		&\left\langle \sum_{i=1}^{N_m}A^*_i\left(A_{i, h}-A_i\right)\left(\hat{f}^\gamma_h-f^\gamma\right), f^\gamma-\hat{f}^\gamma_h\right\rangle_{L^2(0, T)}\leq ch^2\left\| f^\gamma-\hat{f}^\gamma_h\right\|^2_{L^2(Q)},\\
		&\left\langle \sum_{i=1}^{N_m}A^*_iA_i\left(\hat{f}^\gamma_h-f^\gamma\right), f^\gamma-\hat{f}^\gamma_h\right\rangle_{L^2(I)}=-\sum_{i=1}^{N_m}\left\|A_i\left(f^\gamma-\hat{f}^\gamma_h\right) \right\|^2_{L^2(0, T)}<0.
	\end{align*}
	The theorem is proved.
\end{proof}
\begin{cy}\label{cy3.1}
	Let $f^\gamma$ and $\hat{f}^\gamma_h$ are the solution of variational problems \eqref{3.1} and \eqref{3.5}, respectively. Then there hold a error estimate
	\begin{align}\label{3.12}
		\left\|f^\gamma-\hat{f}^{\gamma, \delta}_h \right\|_{L^2(Q)}\leq c(h^2+\delta).
	\end{align}
\end{cy}


\section{Numerical results}
In all examples in this section, we choose the domain $\Omega=(0, 1)\times(0, 1),\, T=1$ and $a_{ij}(x, t)=\delta_{ij}$.
For the temperature we take the exact solution be given by
$$u(x, t)=e^t(x_1-x_1^2)\sin(\pi x_2).$$
We would like to reconstruct function $f$ with several forms of $F$ following
\begin{itemize}
	\item Example 1: $F(x, t)=f(x, t)q(x, t)+g(x, t)$ for IP1 (Example 1.1) with $f(x, t)=\sin(\pi x_1)(x_2-x_2^2)(t^2+1)$,
	\item Example 2: $F(x, t)=f(x)q(x, t)+g(x, t)$ for IP2 (Example 2.1) and IP3 (Example 2.2) with $f(x)=\sin(\pi x_1)(x_2-x_2^2)$,
	\item Example 3: $F(x, t)=f(t)q(x, t)+g(x, t)$ for IP3 with following functions
	\begin{itemize}
		\item[1.]$f(t)=
		\begin{cases}
		2t, & t\in [0, 0.5],\\
		2(1-t), & t \in [0.5, 1],
		\end{cases}$ \qquad for Example 3.1
		\item[2.] $f(t)=
		\begin{cases}
		1, & t\in [0.25, 0.75],\\
		0, & t \notin [0.25, 0.75],
		\end{cases}$ \qquad for Example 3.2
	\end{itemize}
\end{itemize}
We use a uniform decomposition of the domain $Q$ into $65^3=274,625$ nodes and $6\times 64^3=1,572,864$ finite elements. We take $q(x, t)=x_1x_2+t+1$, initial guess $f^*=0, \gamma=10^{-5}$ and level noise $\delta =1\%$.

\newpage
\noindent\textbf{Example 1.1}
\\
We reconstruct $f(x, t)$ with observation in the whole domain.
\begin{figure}[h!]
	\centering
	\includegraphics[width=.5\linewidth]{../Csharp/Results/HS_Q_fxt}
	\caption{The exact $f(x_p, t),\; x_p=(0.5, 0.5)$ and the numerical solution of Example 1.1.}
\end{figure}

\noindent\textbf{Example 2.1}
\\
We reconstruct $f(x)$ with observation is the final overdetermination.
\begin{figure}[h!]
	\begin{subfigure}{.5\linewidth}
		\centering
		\includegraphics[width=\linewidth]{../Csharp/Results/HS_QT_fx1}
	\end{subfigure}%
	\begin{subfigure}{.5\linewidth}
		\centering
		\includegraphics[width=\linewidth]{../Csharp/Results/HS_QT_fx2}
	\end{subfigure}
	\caption{The exacts $f(x_1, 0.5),\; f(0.5, x_2)$ and the numerical solutions of Example 2.1.}
\end{figure}

\newpage
\noindent\textbf{Example 3.1 and 3.2}
\\
We reconstruct $f(t)$ with an integral observation $w(x)=x_1^2+x_2^2+1$ or a point observation $x_0=(0.48, 0.48).$
\begin{figure}[h!]
	\begin{subfigure}{.5\linewidth}
		\centering
		\includegraphics[width=\linewidth]{../Csharp/Results/HS_Integration_ft1}
	\end{subfigure}%
	\begin{subfigure}{.5\linewidth}
		\centering
		\includegraphics[width=\linewidth]{../Csharp/Results/HS_1Point_ft1}
	\end{subfigure}
	\caption{The exact and numerical solution of Example 3.1: integral observation (left) and point observation (right).}
\end{figure}
\begin{figure}[h!]
	\begin{subfigure}{.5\linewidth}
		\centering
		\includegraphics[width=\linewidth]{../Csharp/Results/HS_Integration_ft2}
	\end{subfigure}%
	\begin{subfigure}{.5\linewidth}
		\centering
		\includegraphics[width=\linewidth]{../Csharp/Results/HS_1Point_ft2}
	\end{subfigure}
	\caption{The exact and numerical solution of Example 3.2: integral observation (left) and point observation (right).}
\end{figure}

\newpage
\noindent\textbf{Example 2.2}
\\
We reconstruct $f(x)$ with 9 points described as follows
\begin{figure}[h!]
	\centering
	\includegraphics[width=.5\linewidth]{../Csharp/Results/Nine_Points}
	\caption{fff}
\end{figure}
\begin{figure}[h!]
	\begin{subfigure}{.5\linewidth}
		\centering
		\includegraphics[width=\linewidth]{../Csharp/Results/HS_9Points_fx1}
	\end{subfigure}%
	\begin{subfigure}{.5\linewidth}
		\centering
		\includegraphics[width=\linewidth]{../Csharp/Results/HS_9Points_fx2}
	\end{subfigure}
	\caption{Observation points (left) and the exact $f(x_p, t),\;x_p=(0.5, 0.5)$ and numerical solution of Example 2.2 (right).}
\end{figure}

\justifying
\section{Conclusion}

\newpage
\bibliography{references}{}
\bibliographystyle{plain}

\end{document}
